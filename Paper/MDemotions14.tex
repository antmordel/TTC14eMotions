\newif\ifdraft
\drafttrue
%\draftfalse

\ifdraft
  \documentclass[draft]{llncs}
  \usepackage{color}
  \usepackage[normalem]{ulem}
  \definecolor{green}{rgb}{.2,.8,0}
  \definecolor{blue}{rgb}{0,0,1}
  \definecolor{red}{rgb}{1,0,0}

  \usepackage[final]{graphicx}  
  
  \newcommand{\change}[1]{\textcolor{green}{#1}}
  \newcommand{\delete}[1]{\textcolor{red}{\sout{#1}}}
  \newcommand{\tbc}[1]{\textcolor{blue}{#1}}
  \newcommand{\todo}[1]{\textbf{\color{red}{TO-DO}}: #1}
  \newcommand{\mnote}[1]{\marginnote{#1}}
  
\else
  \documentclass{llncs}
  \usepackage{graphicx}
  
  \newcommand{\change}[1]{#1}
  \newcommand{\delete}[1]{}
  \newcommand{\tbc}[1]{#1}
  \newcommand{\todo}[1]{}
  \newcommand{\mnote}[1]{}
\fi

\usepackage{url}
\usepackage{alltt,verbatim}
\usepackage{soul}
\usepackage{subfig}
\usepackage{pifont}
\usepackage[utf8]{inputenc}

\usepackage[all]{xy}
\newcommand{\code}[1]{{\texttt{#1}}}

% MARGIN NOTES
\marginparwidth 1.25 true in

\newcounter{marginalnote}
\setcounter{marginalnote}{1}
\renewcommand{\themarginalnote}{\roman{marginalnote}}
\newcommand{\marginnote}[1]
           {\raisebox{1ex}{\scriptsize (\themarginalnote)}%
            \marginpar{\footnotesize\raggedright\indent
                       \raisebox{1ex}{\scriptsize (\themarginalnote)} #1}%
            \addtocounter{marginalnote}{1}}
% MARGIN NOTES

% Maude code
\usepackage[final]{listings}
\usepackage{xcolor}
\lstdefinelanguage{maude}
{
	alsoletter={\:},
    morecomment=[l]{---},
    morecomment=[l]{***},
    keywords={pr, protecting, sort, sorts, subsort, subsorts, including, class, msg, msgs, endfm, fmod, is, mod, endm, omod, endom},
    keywords=[2]{eq,  mb, ceq, if, rl, crl, else, then, fi},
    keywords=[3]{ctor, assoc, comm, gather, id\:},
    keywords=[4]{op, ops, var, vars}
}

\lstset{
  language=maude,
  basicstyle=\ttfamily\footnotesize,
  frame=tb,
	framerule=0.2pt,
  %keywordstyle=\bfseries,
	%keywordstyle=[2]\bfseries,
	%keywordstyle=[3]\bfseries,
	%keywordstyle=[4]\bfseries,
}
% Maude code

\DeclareGraphicsExtensions{.png}

\pagestyle{headings} % switches on printing of running heads
%\addtocmark{XXXX} % additional mark in the TOC

\title{The Movie Database Case: \\ A solution using the Maude-based e-Motions tool}
\titlerunning{TO-DO} % abbreviated title (for running head) also used for the TOC unless \toctitle is used

\author{Antonio~Moreno-Delgado \and Francisco~Dur\'an}
\authorrunning{Moreno et al.} %abbreviated author list (for running head)

%%%% modified list of authors for the TOC (add the affiliations)
\tocauthor{
  Antonio Moreno-Delgado (Universidad de M\'alaga),
  Francisco Dur\'an (Universidad de M\'alaga)
}

\institute{
    University of M\'alaga\\
    \email{\{amoreno,duran\}@lcc.uma.es}
    }

\begin{document}

\maketitle

\begin{abstract}
The paper presents solutions for the TTC 2014 Movie Database Case, both in the e-Motions DSML and in the rewriting-logic formal language Maude. The DSLs defined in e-Motions are automatically transformed into Maude specifications, which are then used for simulation and analysis purposes. However, since e-Motions is a general purpose language, in which real-time languages may be modeled, with full support for OCL and other advanced features, the Maude specifications automatically generated are not as efficient as one would like. Since most of these features are not needed for the current tasks, we propose solutions both in e-Motions and in Maude, trying to highlight the main features of both languages. The fact that the solutions given directly in Maude lack the overhead included by e-Motions to deal with all those features it provides that are not needed in the current case study, what makes the solutions given much more efficient, and being able to deal with bigger problems. 
\end{abstract}

%-------------------------------------------------------------
%  INTRODUCTION
%-------------------------------------------------------------
\section{Introduction}
\label{sec:intro}
%-------------------------------------------------------------
%  INTRODUCTION
%-------------------------------------------------------------

\todo{two sentences introducing Maude}

Maude may be seen as a general framework where to develop model transformations. In this way, some work has been done~\cite{TroyaV10}. Since a term is very general, one could specify graphs or models as terms. Thus, a Maude module can define a \textit{in-place} model transformation, where rewriting rules define transitions between two states or models.

e-Motions~\cite{RiveraDV10} is a Domain-Specific Modeling Language (DSML) and a very general tool that supports the specification and simulation of any real-time DSML. Artifacts developed in e-Motions are translated to Maude in a transparent way. The e-Motions simulation is achieved using the Maude engine. Therefore, e-Motions can be seen as a framework where graphically code in Maude.

\subsection{e-Motions}\label{sub:emotions}
For the sake of comprehension of the rest of the paper, in this section we briefly present e-Motions. The definition of a Domain-Specific Language (DSL) typically comprises three tasks: (i) the definition of its abstract syntax, (ii) the definition of its concrete syntax and (iii) the specification of its behavior.

In e-Motions the abstract syntax is defined by means of a Ecore metamodel, in which all the language concepts and the relations between them are specified. The concrete syntax is provided by defining the so-called Graphical Concrete Syntax (GCS). A GCS is a model (conforms the GCS metamodel) where an image is attached to each concept defined in the abstract syntax.

In e-Motions the behavior of a DSL is specified using visual graph-transformation rules. An e-Motions rule consists of a---possibly conditional---Left-Hand Side (LHS), a Right-Hand Side (RHS) and zero or more Negative Application Conditions (NACs). The LHS defines a (sub)-graph matching, optionally conditional. The RHS specifies a (sub)-graph replacement, which if the rule is applied, every object in the LHS that is not in the RHS is deleted, new objects in the RHS that are not in the LHS are created, and those objects whose attributes (or links) are changed are updated. NACs specify conditions or (sub)-graphs such that if there is a matching, the rule cannot be fired.

Fig.~\ref{fig:assemble} shows an example of an e-Motions rule.\footnote{System documentation and several examples are available at \url{http://atenea.lcc.uma.es/e-Motions}.} The objects in both the RHS and LHS are represented by their images defined in the GCS model. Rule \code{Assemble}'s LHS defines the precondition of the rule. It models a assemble machine who needs both a head and a handle in its connected conveyor. If the \code{NAC1}, stating that the current matched \code{Assemble} has not unfinished other rule, is not satisfied, the rule can be applied. The rule is applied as follows. All objects in the LHS which they do not appear in the RHS are deleted, i.e. \code{he} and \code{ha} objects. Those objects in the RHS which do not appear in the LHS are created, setting their attributes properly, i.e. the \code{ham} object with its three attributes. The rest of objects remain changeless. Moreover, as e-Motions is a framework where to define real-time systems, each rule is applied in a established time, i.e. \code{[prodTime,prodTime]} in the \code{Assemble} rule. A rule may contain zero or more local or auxiliary variables. All attribute or variable assignments and conditions are expressed using Object-Constraint Language (OCL)~\cite{ocl}.

The abstract and concrete syntax, and the behavior of a DSL are models and the e-Motions tool has been developed following MDE principles. The Maude code corresponding to a system defined in e-Motions is generated by an ATL/TCS transformation~\cite{atl}.

\begin{figure}[htp]
  \centering
  \includegraphics[width=\textwidth]{imgs/assemble}
  \caption{e-Motions \code{Assemble} rule.}\label{fig:assemble}
\end{figure}



%-------------------------------------------------------------
% SOLUTIONS
%-------------------------------------------------------------
\section{Solution}
\label{sec:solution}
%-------------------------------------------------------------
% SOLUTIONS
%-------------------------------------------------------------

We are presenting two solutions for the different tasks, one graphical solution using e-Motions, and another one using directly Maude. Each task is solved by defining respective DSLs, which share their abstract and concrete syntaxes. The abstract syntax used is the one provided at~\cite{imdbsources} --- we will see below that some of the tasks have required extensions of this common syntax. The main differences between the DSLs defined for the different tasks is in their  concrete behaviors describing what need to be done in each case, that is, the rewrite rules defining the behavior depends on the concrete task and its solution. 

The e-Motions description of the different tasks is then transformed into a Maude specification and executed in Maude. We show how the formal tools available in Maude allow us to check the transformations carried out. Specifically, we will illustrate the use of Maude's reachability analysis capabilities to check that no undesired situation is reached along the execution. 

Although the expressiveness of e-Motions is very welcome in complex problems, thanks to its capabilities to express problems visually, very intuitively and in a language very close to the problem domain, the overhead to be paid in cases like the ones at hand is too high. Specifically, the generality provided by its support for OCL expressions and time requirements, makes that the Maude code generated by the e-Motions tool is not as time performant as we would like. However, the general purpose rewrite-modulo engine at the core of Maude may also be used as a transformation language. Thus, together with the e-Motions solution we present an optimized Maude solution for each task.  

As we will see below, the Maude version of the transformation closely follows the transformations provided in e-Motions, were all rewrite rules are \textit{instantaneous} and expressions are solved directly by Maude built-in types instead of by the OCL interpreter \cite{Roldan-Duran:2008-tr}. Indeed, for problems as simple as the ones at hand, we will see that the representation distance between Maude and e-Motions to the problem domain would be very small, making both solutions very appropriate. Although a more in depth analysis of the problem at hand would most probably have allowed us to even improve the numbers obtained, we have preferred to keep the specification clear and intuitive.



\subsection{Task 1}
\label{sub:task1}
%%%%%%%%%%%%%%%%%%%%%%%%%%%%%%%%%%%%%%%%%%%%%%%%%%%%%%%%%%%%%%%
%\subsection{Task 1}
%\label{sub:task1}
%%%%%%%%%%%%%%%%%%%%%%%%%%%%%%%%%%%%%%%%%%%%%%%%%%%%%%%%%%%%%%%

Task 1 comprises the generation of synthetic models (conforming the movie database metamodel~\cite{imdbcase}) from an input parameter $N \geq 0$. In the following we present an e-Motions based solution and a Maude solution. 


%%%%%%%%%%%%%%%%%%%%%%%%%%%%%%%%%%%%%%%%%%%%%%%%%%%%%%%%%%%%%%%
\subsubsection{e-Motions-based solution.}

Following an e-Motions based approach, we define the abstract and concrete syntax and the behavior of our so-called \textit{Task 1 DSL}. Taking a parameter $N$ as input model, \textit{Task 1 DSL} generates a model containing synthetic data.

As it has been introduced in Section~\ref{sub:emotions}, the abstract syntax of a DSL is given in e-Motions by means of a Ecore metamodel. In fact, this metamodel is provided beforehand in~\cite{imdbsources}. We call this metamodel \textit{Movies MM}. However, the \textit{parameter N} has to be modeled in some way, since in e-Motions the state is just a model. Hence, a new concept call \code{Parameter} has been added to Movies MM. This results in a so-called \textit{Movies* MM}. The class \code{Parameter} has two integer attributes \code{nP} and \code{nN}, positive graphs and negative graphs, due to data generation following Henshing graphs~\cite{henshing} is divided into positive and negative cases.

For the concrete syntax, Fig.~\ref{fig:concreteSyntax} shows how an image has been attached to each concept modeled in the Movies* MM. The behavior of this \textit{Task 1 DSL} is given by means of two in-place rules: \code{createPositive} and \code{createNegative}. Figure~\ref{fig:createPositive} shows the \code{createPositive} rule, which takes an object \code{p} of type \textit{Parameter} with \textit{nP} attribute greater or equal than $0$ and, after the rule application, synthetic data conforming to the Henshin rules~\cite{henshing} are created. Fig.~\ref{fig:createNegative} shows the \code{createNegative} rule, which is analogously defined.

\begin{figure}[htp]
  \subfloat[Actor.\label{fig:actor}]{
    \makebox[60px][c]{\includegraphics[scale=1]{imgs/actor}}
  }
  \hfill
  \subfloat[Actress.\label{fig:actress}]{
    \makebox[60px][c]{\includegraphics[scale=1]{imgs/actress}}
  }
  \hfill
  \subfloat[Movie.\label{fig:movie}]{
    \makebox[60px][c]{\includegraphics[scale=1]{imgs/movie}}
  }
  \hfill
  \subfloat[Couple.\label{fig:couple}]{
    \makebox[60px][c]{\includegraphics[scale=1]{imgs/couple}}
  }
  \hfill
  \subfloat[Parameter.\label{fig:parameter}]{
    \makebox[60px][c]{\includegraphics[scale=1]{imgs/parameter}}
  }
  \caption{Concrete syntax for \textit{Movies* MM}.}
  \label{fig:concreteSyntax}
\end{figure}

Once the syntax and the behavior of the system has been coded, the user may specify a model, which conforms to \textit{Movies* MM}, containing an object \code{Parameter} with its two attributes \code{nP} and \code{nN} properly set. This model is used as initial model of the execution.

This solution is really close to the problem specification~\cite{imdbcase}. Both Fig.~\ref{fig:task1} and Fig.~2 in~\cite{imdbcase} specifying the data generation are almost the same. This solution demonstrates how close is the solution to the problem domain, and how user-friendly is e-Motions.

\subsubsection{Maude version.}
This proposal of Task 1 consists of a object-based Maude specification, which is composed by two main modules: the \code{MOVIES@MM} module defining the classes structure and the \code{TASK1} module defining the solution. The solution is coded using again two rules: \code{createPositive} and \code{createNegative}. One could realized that the Maude version is very much like the e-Motions version. In fact, the former is almost the textual version of the latter. Listing~\ref{lst:createPositive} shows the \code{createPositive} Maude rule that takes the \code{createPositive(N:Nat)} message and a \code{freshOid} auxiliary message---used to create new object identifiers---and returns such a object configuration conforming the Henshin specification~\cite{imdbcase}.

The messages \code{createPositive} and \code{createNegative} are generated in zero-rewrite steps with the equation showed in Listing~\ref{lst:createexampleeq}.

\begin{lstlisting}[caption=\code{createPositive} Maude rule., label=lst:createPositive]
rl [createPositive] :
  createPositive(N)
  freshOid(N')
=>
  < N'     : Movie | rating : (10.0 * float(N)) >
  < N' + 1 : Movie | rating : (10.0 * float(N) + 1.0) >
  < N' + 2 : Movie | rating : (10.0 * float(N) + 2.0) >
  < N' + 3 : Movie | rating : (10.0 * float(N) + 3.0) >
  < N' + 4 : Movie | rating : (10.0 * float(N) + 4.0) >
  
  < N' + 5 : Actor | name : ("a" + string((10 * N),     10)),
                   movies : (N', N' + 1, N' + 2, N' + 3)     >
  < N' + 6 : Actor | name : ("a" + string((10 * N + 1), 10)),
                   movies : (N', N' + 1, N' + 2)             >
  < N' + 7 : Actor | name : ("a" + string((10 * N + 2), 10)),
                   movies : (N' + 1, N' + 2, N' + 3)         >
  < N' + 8 : Actress | name : ("a" + string((10 * N + 3), 10)),
                   movies : (N' + 1, N' + 2, N' + 3, N' + 4) >
  < N' + 9 : Actress | name : ("a" + string((10 * N + 4), 10)),
                   movies : (N' + 1, N' + 2, N' + 3, N' + 4) >
  freshOid(N' + 10) .
\end{lstlisting}

\begin{lstlisting}[label=lst:createexampleeq, caption=Equation \code{createExample(N:Nat)}.]
eq createExample(0) = none .
eq createExample(s(N)) = createPositive(N)
                         createNegative(N)
                         createExample(N) .
\end{lstlisting}

Maude provides a whole formal environment where we can perform proofs of correctness of our solution. For instance, following the results given in~\cite{imdbcase}, \code{createExample($N$)} creates $20N$ objects: $10N$ movies, $5N$ actresses and $5N$ actors. The Maude \code{search} command allows to explore the whole reachable state. A starting term is rewritten a number of steps given, and optionally, a condition can be evaluated in the reached term.

For example, given the \code{numOfMovies} operation which takes an object configuration as input and returns the number of movies in it, we may look for those final states in which the number of moves will be different than $10N$ (being N the parameter of the operation \code{createExample}):
\begin{verbatim}
  search createExample(1) freshOid(0) =>! C:Configuration
    such that numOfMovies(C:Configuration) =/= 10 * 1 .
\end{verbatim}

The arrow \code{=>!} means that it rewrites the initial term to a final term. Maude returns no solution for the above code, that means all final states reached have exactly 10 movies:
\begin{verbatim}
  No solution.
  states: 5  rewrites: 180 in 0ms cpu (0ms real)
\end{verbatim}

\todo{mejorar el aspecto de las tablas}
\begin{table}
  \begin{center}
	\begin{tabular}{r r r}
	$N$ & Time (s) & \# Rewrites \\
	\hline
	2 & 0.004 & 4910 \\
	10 & 0.016 & 24334 \\
	20 & 0.036 & 48614 \\
	100 & 0.604 & 242854 \\
	1000 & 55.747 & 2428054 \\
	2000 & 395.016 & 4856054 \\
	\hline \\
	\end{tabular}
	\caption{e-Motions times for Task 1.}\label{table:emotionstask1}
	\end{center}
\end{table}

\begin{table}
  \begin{center}
	\begin{tabular}{r r r}
	$N$ & Time (s) & \# Rewrites \\
	\hline
	1 & 0.0 & 68 \\
	1000 & 1.908 & 67001 \\
	2000 & 12.740 & 134001 \\
	3000 & 33.866 & 201001 \\
	4000 & 64.060 & 268001 \\
	5000 & 104.442 & 335001 \\
	6000 & 109.614 & 402001 \\
	7000 & 144.521 & 469001 \\
	8000 & 197.340 & 536001 \\
	9000 & 256.984 & 603001 \\
	10000 & 318.271 & 670001 \\
	\hline \\
	\end{tabular}
	\caption{Maude times for Task 1.}\label{table:maudetask1}
	\end{center}
\end{table}

\begin{figure}[htp]
  \subfloat[The \code{createPositive} rule.\label{fig:createPositive}]{%
    \includegraphics[width=\textheight, angle=90]{imgs/createPositiveRule}
  }
  \hfill
%  \subfloat[Rules' headers.]{
%    \includegraphics[width=0.2\textwidth]{imgs/headersCreate}
%  }
%  \hfill
  \subfloat[The \code{negativePositive} rule.\label{fig:createNegative}]{%
    \includegraphics[width=\textheight, angle=90]{imgs/createNegativeRule}
  }
 
  \caption{Task 1 rules. \label{fig:task1}}
\end{figure}




\subsection{Task 2}
\label{sub:task2}
%%%%%%%%%%%%%%%%%%%%%%%%%%%%%%%%%%%%%%%%%%%%%%%%%%%%%%%%%%%%%%%
%\subsection{Task 2}
%\label{sub:task2}

Task 2 consists of extract all couples from a given model, either from Task 1 or IMBd database~\cite{imdbsources}. Two persons are couple whether they played together in at least three movies~\cite{imdbcase}. Once again we present the e-Motions and the Maude solution.

\subsubsection{e-Motions-based solution}

Fig.~\ref{fig:createCouple} shows the \code{createCouple} rule which implements the whole task. \code{Person} objects are shown using square shapes because \code{Person} is an abstract class and it does not have image attached. The \code{createCouple} rule consists of a LHS with a OCL condition, which states \textit{``LHS holds iff there are two persons \code{per1} and \code{per2}, such that the number of movies in the intersection between \code{per1}'s movies and \code{per2}'s movies is greater or equal than 3''}. Moreover, the \code{coupleHasNotBeenCreated} NAC avoids the application of the rule if the couple already exists. 

However, although this solution works, one can notice that the number of matchings in the LHS of the rule is combinatorial. The fact that there are such a large number of matchings makes this solution too inefficient. \todo{add numbers supporting this.}


\begin{figure}[htp]
  \centering
  \includegraphics[width=\textwidth]{imgs/ruleCouples}
  \caption{\code{createCouple} rule.}\label{fig:createCouple}
\end{figure}

We have implemented another solution in which we limit the number of matchings using the next algorithm:
\begin{enumerate}
  \item We split \code{Person}s and \code{Movie}s into separate configurations.
  \item We fix a \code{Person}.
  \item Given a \code{Person}, we look for all couples.
  \item Whether the current \code{Person} set has been gone over all the other persons, we set the next person, and the current person is move to the resulting collection.
\end{enumerate}

Following this approach the number of persons to match as possible couple decrease. A new concept so-called \code{Collection} is added to the metamodel with its concrete syntax to implement the algorithm above mentioned. Fig.~\ref{fig:areCouples} shows one of the rules specified for this solution,\footnote{The rest of the rules are available at \url{https://github.com/antmordel/ttc14emotions}.} namely those that creates new couples. The box is the concrete syntax for the \code{Collection} concept.

\begin{figure}[htp]
  \centering
  \includegraphics[width=\textwidth]{imgs/initialRule}
  \caption{\code{initialRule} rule.}\label{fig:initialRule}
\end{figure}

\begin{figure}[htp]
  \centering
  \includegraphics[width=\textwidth]{imgs/fixPerson}
  \caption{\code{fixPerson} rule.}\label{fig:fixPerson}
\end{figure}

\begin{figure}[htp]
  \centering
  \includegraphics[width=\textwidth]{imgs/areCouple}
  \caption{\code{doingCouples} rule.}\label{fig:areCouples}
\end{figure}

\begin{figure}[htp]
  \centering
  \includegraphics[width=\textwidth]{imgs/areNotCouple}
  \caption{\code{doingCouples} rule.}\label{fig:areNotCouples}
\end{figure}

\begin{figure}[htp]
  \centering
  \includegraphics[width=\textwidth]{imgs/nextPerson}
  \caption{\code{nextPerson} rule.}\label{fig:nextPerson}
\end{figure}

\todo{add numbers}

\subsubsection{Maude-based solution}



\subsection{Task 3}
\label{sub:task3}
%%%%%%%%%%%%%%%%%%%%%%%%%%%%%%%%%%%%%%%%%%%%%%%%%%%%%%%%%%
%% Task 3
%%%%%%%%%%%%%%%%%%%%%%%%%%%%%%%%%%%%%%%%%%%%%%%%%%%%%%%%%%

Given a model with couples already created, Task 3 consists in calculating the average rating of shared movies for each of these couples. As for the previous tasks, we provide solutions both in e-Motions and Maude. 

%%%%%%%%%%%%%%%%%%%%%%%%%%%%%%%%%%%%%%%%%%%%%%%%%%%%%%%%%%%%%%%
\subsubsection{e-Motions-based solution.}

The solution consists in one single rule, shown in Figure~\ref{fig:computingAvgRating}, in which the average is calculated only once for each couple. Notice the use of an action in the NAC of the rule to state that the value has not been already calculated. The number of rewrites and execution times for $N=2, 10$ are shown in Table~\ref{table:emotionstask3}.

\begin{figure}[htp]
  \centering
  \includegraphics[width=\textwidth]{imgs/computingAvgRating}
  \caption{\code{computingAvgRating} rule.}\label{fig:computingAvgRating}
\end{figure}

\begin{table*}[tb]
\renewcommand{\tabcolsep}{6pt}
\renewcommand{\arraystretch}{1.2}
    \centering
	\begin{tabular}{r r r}
	$N$ & Time (s) & \# Rewrites \\
	\hline
	2 & 0.0 & 4,527 \\
	10 & 2.1 & 891,432 \\
	\hline \\
	\end{tabular}
	\caption{e-Motions times for Task 3.}\label{table:emotionstask3}
\end{table*}

%%%%%%%%%%%%%%%%%%%%%%%%%%%%%%%%%%%%%%%%%%%%%%%%%%%%%%%%%%%%%%%
\subsubsection{Maude-based solution.}
The corresponding Maude rule specifying the solution of this task is shown in Listing~\ref{lst:task3}. Table~\ref{table:maudetask3} show the number of rewrites and execution times for problems of sizes $100$, $200$, $300$, and $400$.

\begin{lstlisting}[caption=Maude rule for Task 3 solution., label=lst:task3]
crl [avgRating] :
  { < M : Couple | commonMovies : MovieSet, 
       avgRating : 0.0, 
       Atts1 >
    couplesCalculated(Couples) 
    C 
  }
=>
  { < M : Couple | commonMovies : MovieSet, 
       avgRating : sumAllRatings(MovieSet, C) 
                     / float(| MovieSet |),
       Atts1 >
    couplesCalculated((M, Couples)) 
    C
  }
if not(M in Couples) .
\end{lstlisting}

\begin{table}
  \begin{center}
	\begin{tabular}{r r r}
	$N$ & Time (s) & \# Rewrites \\
	\hline
	100 & 1.5 & 21,800 \\
	200 & 6.3 & 43,600 \\
	300 & 14.9 & 65,400 \\
	400 & 29.7 & 87,200 \\
	\hline \\
	\end{tabular}
	\caption{Maude times for Task 3.}\label{table:maudetask3}
	\end{center}
\end{table}

%-------------------------------------------------------------
% CONCLUSIONS
%-------------------------------------------------------------
\section{Conclusions}

%-------------------------------------------------------------
% CONCLUSIONS
%-------------------------------------------------------------
%\section{Conclusions}

We have presented solutions for the TTC 2014 Movie Database Case both in the e-Motions DSML and in the rewriting-logic formal language Maude. 

e-Motions provides a very rich set of features, that enables the formal and precise definition of real-time DSMLs as models in a graphical and intuitive way. It makes use of an extension of in-place model transformation with a model of timed behavior and a mechanism to state action properties. The extension is defined in such a way that it avoids artificially modifying the DSML's metamodel to include time and action properties. Moreover, it supports attribute computations and ordered collections, which are specified by means of OCL
expressions. All these features makes the language very expressive, but directly impact on performance. 

The Maude solutions presented are also very intuitive and simple. The fact that the solutions given directly in Maude lack the overhead included by \mbox{e-Motions} to deal with all those features it provides that are not needed in the current case study, makes the solutions given much more efficient, and able to deal with bigger problems. 

We believe that the Maude solutions could still be made more efficient, but possibly at the cost of loosing elegance in the solutions. 

\bibliographystyle{splncs03}
\bibliography{TTC14,duran}
\providecommand{\url}[1]{\texttt{#1}}
\end{document}
