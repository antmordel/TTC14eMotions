%-------------------------------------------------------------
% SOLUTIONS
%-------------------------------------------------------------

We are presenting two solutions for the different tasks, one graphical solution using e-Motions, and another one using directly Maude. Each task is solved by defining respective DSLs, which share their abstract and concrete syntaxes. The abstract syntax used is the one provided at~\cite{imdbsources} --- we will see below that some of the tasks have required extensions of this common syntax. The main differences between the DSLs defined for the different tasks is in their  concrete behaviors describing what need to be done in each case, that is, the rewrite rules defining the behavior depends on the concrete task and its solution. 

The e-Motions description of the different tasks is then transformed into a Maude specification and executed in Maude. We show how the formal tools available in Maude allow us to check the transformations carried out. Specifically, we will illustrate the use of Maude's reachability analysis capabilities to check that no undesired situation is reached along the execution. 

Although the expressiveness of e-Motions is very welcome in complex problems, thanks to its capabilities to express problems visually, very intuitively and in a language very close to the problem domain, the overhead to be paid in cases like the ones at hand is too high. Specifically, the generality provided by its support for OCL expressions and time requirements, makes that the Maude code generated by the e-Motions tool is not as time performant as we would like. However, the general purpose rewrite-modulo engine at the core of Maude may also be used as a transformation language. Thus, together with the e-Motions solution we present an optimized Maude solution for each task.  

As we will see below, the Maude version of the transformation closely follows the transformations provided in e-Motions, were all rewrite rules are \textit{instantaneous} and expressions are solved directly by Maude built-in types instead of by the OCL interpreter \cite{Roldan-Duran:2008-tr}. Indeed, for problems as simple as the ones at hand, we will see that the representation distance between Maude and e-Motions to the problem domain would be very small, making both solutions very appropriate. Although a more in depth analysis of the problem at hand would most probably have allowed us to even improve the numbers obtained, we have preferred to keep the specification clear and intuitive.

