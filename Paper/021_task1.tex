%%%%%%%%%%%%%%%%%%%%%%%%%%%%%%%%%%%%%%%%%%%%%%%%%%%%%%%%%%%%%%%
%\subsection{Task 1}
%\label{sub:task1}
%%%%%%%%%%%%%%%%%%%%%%%%%%%%%%%%%%%%%%%%%%%%%%%%%%%%%%%%%%%%%%%

Task 1 comprises the generation of synthetic models (conforming the movie database metamodel~\cite{imdbcase}) from an input parameter $N \geq 0$. In the following we present an e-Motions based solution and a Maude solution. 


%%%%%%%%%%%%%%%%%%%%%%%%%%%%%%%%%%%%%%%%%%%%%%%%%%%%%%%%%%%%%%%
\subsubsection{e-Motions-based solution.}

Following an e-Motions based approach, we define the abstract and concrete syntax and the behavior of our so-called \textit{Task 1 DSL}. Taking a parameter $N$ as input model, \textit{Task 1 DSL} generates a model containing synthetic data.

As it has been introduced in Section~\ref{sub:emotions}, the abstract syntax of a DSL is given in e-Motions by means of a Ecore metamodel. In fact, this metamodel is provided beforehand in~\cite{imdbsources}. We call this metamodel \textit{Movies MM}. However, the \textit{parameter N} has to be modeled in some way, since in e-Motions the state is just a model. Hence, a new concept call \code{Parameter} has been added to Movies MM. This results in a so-called \textit{Movies* MM}. The class \code{Parameter} has two integer attributes \code{nP} and \code{nN}, positive graphs and negative graphs, due to data generation following Henshing graphs~\cite{henshing} is divided into positive and negative cases.

For the concrete syntax, Fig.~\ref{fig:concreteSyntax} shows how an image has been attached to each concept modeled in the Movies* MM. The behavior of this \textit{Task 1 DSL} is given by means of two in-place rules: \code{createPositive} and \code{createNegative}. Figure~\ref{fig:createPositive} shows the \code{createPositive} rule, which takes an object \code{p} of type \textit{Parameter} with \textit{nP} attribute greater or equal than $0$ and, after the rule application, synthetic data conforming to the Henshin rules~\cite{henshing} are created. Fig.~\ref{fig:createNegative} shows the \code{createNegative} rule, which is analogously defined.

\begin{figure}[htp]
  \subfloat[Actor.\label{fig:actor}]{
    \makebox[60px][c]{\includegraphics[scale=1]{imgs/actor}}
  }
  \hfill
  \subfloat[Actress.\label{fig:actress}]{
    \makebox[60px][c]{\includegraphics[scale=1]{imgs/actress}}
  }
  \hfill
  \subfloat[Movie.\label{fig:movie}]{
    \makebox[60px][c]{\includegraphics[scale=1]{imgs/movie}}
  }
  \hfill
  \subfloat[Couple.\label{fig:couple}]{
    \makebox[60px][c]{\includegraphics[scale=1]{imgs/couple}}
  }
  \hfill
  \subfloat[Parameter.\label{fig:parameter}]{
    \makebox[60px][c]{\includegraphics[scale=1]{imgs/parameter}}
  }
  \caption{Concrete syntax for \textit{Movies* MM}.}
  \label{fig:concreteSyntax}
\end{figure}

Once the syntax and the behavior of the system has been coded, the user may specify a model, which conforms to \textit{Movies* MM}, containing an object \code{Parameter} with its two attributes \code{nP} and \code{nN} properly set. This model is used as initial model of the execution.

This solution is really close to the problem specification~\cite{imdbcase}. Both Fig.~\ref{fig:task1} and Fig.~2 in~\cite{imdbcase} specifying the data generation are almost the same. This solution demonstrates how close is the solution to the problem domain, and how user-friendly is e-Motions.

\subsubsection{Maude version.}
This proposal of Task 1 consists of a object-based Maude specification, which is composed by two main modules: the \code{MOVIES@MM} module defining the classes structure and the \code{TASK1} module defining the solution. The solution is coded using again two rules: \code{createPositive} and \code{createNegative}. One could realized that the Maude version is very much like the e-Motions version. In fact, the former is almost the textual version of the latter. Listing~\ref{lst:createPositive} shows the \code{createPositive} Maude rule that takes the \code{createPositive(N:Nat)} message and a \code{freshOid} auxiliary message---used to create new object identifiers---and returns such a object configuration conforming the Henshin specification~\cite{imdbcase}.

The messages \code{createPositive} and \code{createNegative} are generated in zero-rewrite steps with the equation showed in Listing~\ref{lst:createexampleeq}.

\begin{lstlisting}[caption=\code{createPositive} Maude rule., label=lst:createPositive]
rl [createPositive] :
  createPositive(N)
  freshOid(N')
=>
  < N'     : Movie | rating : (10.0 * float(N)) >
  < N' + 1 : Movie | rating : (10.0 * float(N) + 1.0) >
  < N' + 2 : Movie | rating : (10.0 * float(N) + 2.0) >
  < N' + 3 : Movie | rating : (10.0 * float(N) + 3.0) >
  < N' + 4 : Movie | rating : (10.0 * float(N) + 4.0) >
  
  < N' + 5 : Actor | name : ("a" + string((10 * N),     10)),
                   movies : (N', N' + 1, N' + 2, N' + 3)     >
  < N' + 6 : Actor | name : ("a" + string((10 * N + 1), 10)),
                   movies : (N', N' + 1, N' + 2)             >
  < N' + 7 : Actor | name : ("a" + string((10 * N + 2), 10)),
                   movies : (N' + 1, N' + 2, N' + 3)         >
  < N' + 8 : Actress | name : ("a" + string((10 * N + 3), 10)),
                   movies : (N' + 1, N' + 2, N' + 3, N' + 4) >
  < N' + 9 : Actress | name : ("a" + string((10 * N + 4), 10)),
                   movies : (N' + 1, N' + 2, N' + 3, N' + 4) >
  freshOid(N' + 10) .
\end{lstlisting}

\begin{lstlisting}[label=lst:createexampleeq, caption=Equation \code{createExample(N:Nat)}.]
eq createExample(0) = none .
eq createExample(s(N)) = createPositive(N)
                         createNegative(N)
                         createExample(N) .
\end{lstlisting}

Maude provides a whole formal environment where we can perform proofs of correctness of our solution. For instance, following the results given in~\cite{imdbcase}, \code{createExample($N$)} creates $20N$ objects: $10N$ movies, $5N$ actresses and $5N$ actors. The Maude \code{search} command allows to explore the whole reachable state. A starting term is rewritten a number of steps given, and optionally, a condition can be evaluated in the reached term.

For example, given the \code{numOfMovies} operation which takes an object configuration as input and returns the number of movies in it, we may look for those final states in which the number of moves will be different than $10N$ (being N the parameter of the operation \code{createExample}):
\begin{verbatim}
  search createExample(1) freshOid(0) =>! C:Configuration
    such that numOfMovies(C:Configuration) =/= 10 * 1 .
\end{verbatim}

The arrow \code{=>!} means that it rewrites the initial term to a final term. Maude returns no solution for the above code, that means all final states reached have exactly 10 movies:
\begin{verbatim}
  No solution.
  states: 5  rewrites: 180 in 0ms cpu (0ms real)
\end{verbatim}

\todo{mejorar el aspecto de las tablas}
\begin{table}
  \begin{center}
	\begin{tabular}{r r r}
	$N$ & Time (s) & \# Rewrites \\
	\hline
	2 & 0.004 & 4910 \\
	10 & 0.016 & 24334 \\
	20 & 0.036 & 48614 \\
	100 & 0.604 & 242854 \\
	1000 & 55.747 & 2428054 \\
	2000 & 395.016 & 4856054 \\
	\hline \\
	\end{tabular}
	\caption{e-Motions times for Task 1.}\label{table:emotionstask1}
	\end{center}
\end{table}

\begin{table}
  \begin{center}
	\begin{tabular}{r r r}
	$N$ & Time (s) & \# Rewrites \\
	\hline
	1 & 0.0 & 68 \\
	1000 & 1.908 & 67001 \\
	2000 & 12.740 & 134001 \\
	3000 & 33.866 & 201001 \\
	4000 & 64.060 & 268001 \\
	5000 & 104.442 & 335001 \\
	6000 & 109.614 & 402001 \\
	7000 & 144.521 & 469001 \\
	8000 & 197.340 & 536001 \\
	9000 & 256.984 & 603001 \\
	10000 & 318.271 & 670001 \\
	\hline \\
	\end{tabular}
	\caption{Maude times for Task 1.}\label{table:maudetask1}
	\end{center}
\end{table}

\begin{figure}[htp]
  \subfloat[The \code{createPositive} rule.\label{fig:createPositive}]{%
    \includegraphics[width=\textheight, angle=90]{imgs/createPositiveRule}
  }
  \hfill
%  \subfloat[Rules' headers.]{
%    \includegraphics[width=0.2\textwidth]{imgs/headersCreate}
%  }
%  \hfill
  \subfloat[The \code{negativePositive} rule.\label{fig:createNegative}]{%
    \includegraphics[width=\textheight, angle=90]{imgs/createNegativeRule}
  }
 
  \caption{Task 1 rules. \label{fig:task1}}
\end{figure}


