%-------------------------------------------------------------
% CONCLUSIONS
%-------------------------------------------------------------
%\section{Conclusions}

We have presented solutions for the TTC 2014 Movie Database Case both in the e-Motions DSML and in the rewriting-logic formal language Maude. 

e-Motions provides a very rich set of features, that enables the formal and precise definition of real-time DSMLs as models in a graphical and intuitive way. It makes use of an extension of in-place model transformation with a model of timed behavior and a mechanism to state action properties. The extension is defined in such a way that it avoids artificially modifying the DSML's metamodel to include time and action properties. Moreover, it supports attribute computations and ordered collections, which are specified by means of OCL
expressions. All these features makes the language very expressive, but directly impact on performance. 

The Maude solutions presented are also very intuitive and simple. The fact that the solutions given directly in Maude lack the overhead included by e-Motions to deal with all those features it provides that are not needed in the current case study, what makes the solutions given much more efficient, and being able to deal with bigger problems. 

We believe that the Maude solutions could still be made more efficient, but possibly at the cost of loosing elegance in the solutions. 